\begin{savequote}[15cm]
  \vspace{-30mm}
  \raggedleft
\sffamily
``The errors you made and are about to make have already been made by me'' \\said the experienced man to the newbie, selling it as wisdom.
\qauthor{You, in the future}
\end{savequote}

\chapter{Tips and Tricks}

\begin{itemize}
\item Either install \LaTeX on your own machine, maybe in a
  docker-image, or use overleaf. \\ 
    On your own machine, the compilation will commence more
    quickly. There is also no compilation timeout, which you may run
    into with overleaf. 
\item actively use $\backslash$include (for chapters) and $\backslash$input to break you 
 big document into smaller parts. If you then have a file called
 IncludeOnly.tex in your root dir, only the chapters in that file will
 be included. In this way it is easy to make something for your
 reviewers, but also speeds up the compilation process drastically,
 which is nice while you are editing and you work (re)view-driven. 
\item If you add pictures or tables, always choose vector
  format. Never jpeg unless it is a photo, png only for screenshots,
  which you should try to avoid. In the drawing tool (Visual Paradigm,
  Umlet, drawio)  you can most likely export in pdf format which you
  can include with $\backslash$includegraphics. If you have svg, which
  the other popular vector format, you can use a conversion tool to
  turn it into a pdf. Inkscape, which is available on all relevant
  platforms, does the trick for me. 
  For tables, a spreadsheet(excel libreoffice calc) is a reasonable
  choice. Export the selection as pdf with no borders.  Use a tool to
  clip/crop off the white borders. pdfcrop works for me there. Include
  the pdf with $\backslash$includegraphics, but put in in table
  environment. 
  
\end{itemize}
