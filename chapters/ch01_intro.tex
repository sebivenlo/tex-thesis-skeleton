\def\TheFile{ch01_intro.tex}

% quote text
\begin{savequote}[15cm]
  \vspace{-30mm}%
  \raggedleft
  \sffamily
  Introducing oneself properly is always hard.
  \qauthor{Anonymous}
\end{savequote}
\chapter{Introduction}

Writing documentation is often considered a chore. But actually reading student reports is even worse.
Certainly if the student is to wordy, sloppy, repeats every other section and so on.

The Casus Belli, in this case, is that I have been an examiner of a lot,
not to say most students in the informatics courses at Fontys
Hogeschool Venlo. There I have to read 22 reports, from 14 students that
I coach and 8 others where I am the examiner. That incentivised me to
write down some advice. Here you have it.

\section{Do not bore us to death!}

\begin{wrapfigure}{r}{.25\textwidth}
  \includegraphics[width=.25\textwidth]{images/boreme.png}
%  \caption{\label{fig:boredom}Boredom}
\end{wrapfigure}
The worst thing that can happen to me, the poor person that has to
read your whole report from front page until the last page before the
appendix, is that I get the impression that some one is trying to hum
me asleep.  I have no clue what or who caused this, but many students
have the habit to explain in the first paragraph of the chapter what
they are going to tell in that chapter. Why? Why Waste my time and
your time? Do you think that improves your grade? Because I have to
read the whole damn thing anyway, why take the excitement of 
reading away. Would you read a book or watch a film if the first
paragraph of each chapters gives everything away?

\section{Things to avoid}
To keep the reader awake, and more importantly: interested, and appreciative of your work you should:
\begin{Description}
\item[Stay DRY] DO NOT repeat stuff.\gls{DRY}
\item[Good titles] Think of good chapter and section titles. We read
  and use the table of contents. If the chapter and section titles are
  good, they help explain the structure of the report, without any
  extra boring help.\Margin{Good titles}
\item[Be brief] You do not get paid per written word, nor are we paid per word \textbf{read}.\Margin{brief}
\item[Use a storyline] Both in your report and in your
  presentation. If you invent a \define{\gls{persona}} anyway, use him or her as the
  protagonist to tell the store, and use him/her to explain stuff. 
The story need not necessarily be true to the actual chronology of the
time spent on your bachelor project, but should be a logical story, in
which you take the reader along to explain your reasoning, the
decisions you made and why, etc. 
\end{Description}

Your protagonist may not be useful in all the parts, but might be very
handy to explain the problem, and assignment and how he or she can use the
fruits of your labor.

Start with the \textbf{company} and every detail of that company that is relevant to the project.
Then continue with the \textbf{context} of the problem, exactly as
much as you need to explain the next logical thing: the
\textbf{problem} that you have been asked to solve. Followed by the
\textbf{assignment}.

If in any paragraph that you wrote you or a reviewer notices that you
have to explain things that could have been explained in an earlier
part, do not repeat any of said part, not even by saying 'as you
can/have read in ..', but instead revise the earlier part so the
context, the problem, the assignment etc. can be understood in one
flow.

As an example, If you need to explain what a typical customer will do
with the product, then we expect that you have explained the product and
customer in the company description or context.

You should assume that the reader is a single-pass compiler (like
\LaTeX\ a C- or Pascal compiler). What has not been defined before
cannot be used.

Forward references (like: "as you will see in ...") are \textbf{not
  allowed}. It is a waste of words, in particular when the structure,
and thus the table of contents is any good.

In this way, you will keep the reader in his flow because he does not
have to page forward or backward, and if the story is short enough,
the reader will be able to pull through without being bored to death.

Also: By NOT repeating stuff, you have no chance of inconsistencies,
because every paragraph is the only source of its truth.

\section{TLDR;}\gls{TLDR}

In the remainder of this document, you will see some tips and tricks to
use when you write your report in \LaTeX, but the above and some
things in the use of graphics also apply when you use \textbf{Word} or
some other text processing application. The quality of your report
should not suffer from your choice of tools. So read those chapters as
well. In particular when it is about adding pictures, listings, and
tables.


\section{Use a better technology}
One of the standards for documentation in open source and hence in
Linux land is \LaTeX, a text processing package. \LaTeX\ is available
for free and available with all Linux distributions and can be
installed on Windows and Max OSX just as easily.

\TeX\ is the machinery of \LaTeX\ and was defined in the \TeX\ book
\cite{texbook} and implemented by
Prof. Donald Knuth. \LaTeX\ is a (nowadays HUGE) set of macros built
on top of that. \LaTeX\ in its initial form is described by Leslie
Lamport in \cite{latexbook}. If you like your book thick, try the
\LaTeX\ companion \cite{latexcompanion}.

The web is also a very good source of \LaTeX\ documentation. A good
starting point is \url{http://en.wikibooks.org/wiki/LaTeX}, useful for
beginners and pros alike. The help pages on overleaf are also quite good.

This is a simple multi-part document. Its purpose is to show how easy it is
to create a multi-part document, one that, for instance, can be worked on 
simultaneously by several authors. Note that most of the settings for
this document are set in the file \texttt{configuration/thesis\_config.tex}. 
Look in that file too.

You are kindly advised to keep your lab logs in simple text
files. These can be turned into latex files easily,
which can be used to produce a nice-looking report.
%\clearpage 
\section{Some hints to start with}
Sometimes things do not work out the way you think.
\LaTeX\ interprets some character codes in its way.
Things like dollar signs or even underscore are special.
\LaTeX\ sources are littered with accolades or \textit{curly braces} if that's
the way you call them. They are special too. So here is some advice: 

Do not use \define{funny file names}. That is: stick to ASCII filenames without spaces or even underscores. 
These will lead you only into trouble. If you want to keep things portable, 
don't use camel case (like in JavaClassNames) either, because
some OS-es do not distinguish between upper and lower case. You may of
course break this rule if the files are program things like
Java source files.

\subsection{Hints for informatics (use version control)}
In software projects, versioning is important. \LaTeX\ and \textsc{GIT}
work nicely together here.

To keep these version codes up to date, first check if your \LaTeX\  files compile,
then add and commit them and do your final \LaTeX\  run. 

%%% Local Variables: 
%%% mode: latex
%%% TeX-master: "main"
%%% End: 
